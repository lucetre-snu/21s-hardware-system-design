\documentclass{article}
\usepackage{bm}
\usepackage{amsmath}
\usepackage{graphicx}
\usepackage{mdwlist}
\usepackage[colorlinks=true]{hyperref}
\usepackage{geometry}
\usepackage{kotex}
\geometry{margin=1in}
\geometry{headheight=2in}
\geometry{top=2in}
\usepackage{palatino}
%\renewcommand{\rmdefault}{palatino}
\usepackage{fancyhdr}

\newcommand{\red}[1]{{\color{red} #1}}
\newcommand{\blue}[1]{{\color{blue} #1}}
\newcommand{\orange}[1]{{\color{orange} #1}}
\newcommand{\purple}[1]{{\color{purple} #1}}

%\pagestyle{fancy}
\rhead{}
\lhead{}
\chead{%
  {\vbox{%
      \vspace{2mm}
      \large
      Hardware System Design 4190.309A\hfill
\\
      Seoul National University
      \\[4mm]
      \textbf{Practice \#9. Convolution Lowering SW}\\
      \textbf{Jiwon Lee, Sangjun Son}
    }
  }
}

%%%%%%%%%%%%%%%%%%%%%%%
\usepackage{xcolor}
\usepackage{listings}
\definecolor{vgreen}{RGB}{104,180,104}
\definecolor{vblue}{RGB}{49,49,255}
\definecolor{vorange}{RGB}{255,143,102}

\lstdefinestyle{c-style}
{
    language=C,
    basicstyle=\scriptsize\ttfamily,
    keywordstyle=\color{vblue},
    identifierstyle=\color{black},
    commentstyle=\color{vgreen},
    numbers=left,
    numberstyle=\tiny\color{black},
    numbersep=10pt,
    tabsize=8,
    moredelim=*[s][\colorIndex]{[}{]},
    literate=*{:}{:}1
}
\lstdefinestyle{python-style}
{
    language=Python,
    basicstyle=\scriptsize\ttfamily,
    keywordstyle=\color{vblue},
    identifierstyle=\color{black},
    commentstyle=\color{vgreen},
    numbers=left,
    numberstyle=\tiny\color{black},
    numbersep=10pt,
    tabsize=8,
    moredelim=*[s][\colorIndex]{[}{]},
    literate=*{:}{:}1
}

\makeatletter
\newcommand*\@lbracket{[}
\newcommand*\@rbracket{]}
\newcommand*\@colon{:}
\newcommand*\colorIndex{%
    \edef\@temp{\the\lst@token}%
    \ifx\@temp\@lbracket \color{black}%
    \else\ifx\@temp\@rbracket \color{black}%
    \else\ifx\@temp\@colon \color{black}%
    \else \color{vorange}%
    \fi\fi\fi
}
\makeatother

\usepackage{trace}
%%%%%%%%%%%%%%%%%%%%%%%

\usepackage{paralist}

\usepackage{todonotes}
\setlength{\marginparwidth}{2.15cm}

\usepackage{tikz}
\usetikzlibrary{positioning,shapes,backgrounds}

\begin{document}

\pagestyle{fancy}

\section*{Goal}

\begin{itemize*}
\item Implement convolution lowering in C++.
\item Integrate convolution lowering into the pretrained model (CNN).
\begin{itemize*}
\item On MNIST dataset as Figure~\ref{fig1}.
\end{itemize*}
\end{itemize*}
\begin{figure}[ht]
	\centering
	\includegraphics[width=0.4\textwidth]{fig/mnist.png}
\caption{MNIST dataset (Category: digits(0-9), Image size: 28x28, Color: gray scale, Number of images [train/test]: 60000/10000).}
\label{fig1}
\end{figure}

\section{Implementation}
지난 Lab 2에서 구현한 Multi-Layer Perceptron으로 MNIST 숫자를 예측하는 프로젝트의 연장선에서 CNN 모델 또한 구현하기 위해 Matrix-Matrix Multiplication과 Convolution Lowering을 구현하고 이를 Zedboard CPU 위에서 수행하는 실험을 한다. \\

MLP 모델의 경우 28 x 28 크기의 숫자 이미지이며 이를 벡터화 하여 여러 Layer를 거쳐 마지막에 output layer에 10개의 숫자 중 나타날 확률을 학습하게 된다. 행렬과 벡터 곱셈이 매우 크므로 공간이나 시간적 복잡도를 최적화하기 위해서는 Tiling method를 사용하였으며 이 방법을 CNN 모델의 행렬 곱셈에서도 사용하게 된다. 행렬과 벡터 또는 행렬과 행렬을 일정한 크기로 나누어 가속기를 통해 쓰레드를 나누어 빠른 연산 속도를 가능하게 한다. \\

\begin{figure}[htb!]
	\centering
	\includegraphics[width=0.8\textwidth]{fig/overview.jpg}
\caption{CNN 모델에서 사용하는 Operation, Convolution Lowering (Lab 9), Matrix-Matrix Multiplication (Lab 9), Matrix-Vector Multiplication (Lab 2)~\cite{lab2, lab9}.}
\label{fig2}
\end{figure}

결론적으로 이번 실습에서 구현하고 검증해야 하는 Operation은 다음과 같이 세 가지이다.
\begin{enumerate}
    \item \textit{Matrix-Vector Multiplication}\\
    Block operation (Tiling Method)을 활용한 행렬 벡터 연산
    \item \textit{Matrix-Matrix Multiplication}\\
    Block operation (Tiling Method)을 활용한 행렬 행렬 연산
    \item \textit{Convolution Lowering} \\
    Image와 Convolution Filter 데이터를 재정렬하여 Matrix Multiplication로 바꾸는 행렬화 작업~\cite{chellapilla2006high, chetlur2014cudnn}
\end{enumerate}

\subsection{Matrix-Vector Multiplication}
\label{sec:mv}
행렬 벡터 연산을 수행할 때는 작은 크기의 행렬 벡터 연산 (Tiling)으로 축소 대응시켜 계산을 하게 되는데 이 때 자주 사용되는 함수가 block operation이다. 함수를 사용하기 위해서 중간 중간에 \texttt{data\_}라는 데이터 중간 저장소를 통해 곱셈에 사용될 행렬과 벡터를 fetching 하고 연산된 output vector를 덮어쓰는 과정을 반복하게 된다. 코드에 대한 자세한 설명은 Lab 2를 참고하면 된다~\cite{lab2}.

\subsection*{\texttt{FPGA::largeMV}}
\begin{lstlisting}[style={c-style}]
void FPGA::largeMV(const float* large_mat, const float* input, float* output, int num_input, int num_output)
{
  float* vec = this->vector();
  float* mat = this->matrix();

  // 0) Initialize output vector		
  for(int i = 0; i < num_output; ++i)
    output[i] = 0;

  for(int i = 0; i < num_output; i += m_size_)
  {
    for(int j = 0; j < num_input; j += v_size_)
    {			
      // 0) Initialize input vector
      int block_row = min(m_size_, num_output-i);
      int block_col = min(v_size_, num_input-j);
            
      // 1) Assign a vector
      for (int col = 0; col < block_col; col++)
        data_[col] = input[j + col];
      for (int col = block_col; col < v_size_; col++)
        data_[col] = 0;

      // 2) Assign a matrix
      for (int row = 0; row < block_row; row++)
        for (int col = 0; col < block_col; col++)
          data_[(row+1)*v_size_ + col] = large_mat[(i+row)*num_input + (j+col)];


      // 3) Call a function `blockMV() to execute MV multiplication
      const float* ret = this->blockMV();

      // 4) Accumulate intermediate results
      for(int row = 0; row < block_row; ++row)
        output[i + row] += ret[row];
    } 
  }
}
\end{lstlisting}

\subsection{Matrix-Matrix Multiplication}
Section~\ref{sec:mv}과 마찬가지로 Tiling Method으로 축소 대응시켜 계산을 하면 된다. 함수를 사용하기 위해서 중간 중간에 \texttt{data\_M}라는 데이터 중간 저장소를 통해 곱셈에 사용될 2개의 행렬을 fetching 하고 연산된 output vector를 덮어쓰는 과정을 반복하게 된다. 

\begin{figure}[htb!]
	\centering
	\includegraphics[width=0.6\textwidth]{fig/mm.jpg}
\caption{Block operation에 사용되는 행렬의 곱셈 연산 부분, 파란색 행렬을 곱셈하기 위해 Tiling Method으로 쪼개었을 때 연두색 행렬끼리의 곱셈으로 Output Matrix를 채워나갈 수 있다. 위 예시는 쪼개는 단위인 \texttt{v\_size}가 64이다.}
\label{fig3}
\end{figure}

Figure~\ref{fig3}에서 볼 수 있듯이 기본적으로 \texttt{v\_size} 간격으로 작은 Block operation을 수행하지만 행렬의 가로, 세로 크기가 항상 \texttt{v\_size}의 배수가 아니므로 경계부분에서의 예외처리를 위해 Block 사이즈를 나타내는 변수 \texttt{block\_row},  \texttt{block\_col\_1}, \texttt{block\_col\_2}를 도입한다.

\subsection*{\texttt{FPGA::largeMM}}
\begin{lstlisting}[style={c-style}]
void FPGA::largeMM(const float* weight_mat, const float* input_mat, float* output, 
		   int num_input, int num_output, int num_matrix2)
{
  float* m1 = this->matrix_M1();
  float* m2 = this->matrix_M2();

  // 0) Initialize output vector		
  for(int i = 0; i < num_output*num_matrix2; ++i)
    output[i] = 0;

  for(int i = 0; i < num_output; i += v_size_)
  {
    for(int j = 0; j < num_input; j += v_size_)
    {
      for(int k = 0; k < num_matrix2; k += v_size_)
      {
        // 0) Initialize input vector
        int block_row = min(v_size_, num_output-i);
        int block_col_1 = min(v_size_, num_input-j);
        int block_col_2 = min(v_size_, num_matrix2-k);

        // 1) Assign a m1
        for (int row = 0; row < block_row; row++) {
          for (int col = 0; col < block_col_1; col++)
            data_M[row*v_size_ + col] = weight_mat[(i+row)*num_input + (j+col)];
          for (int col = block_col_1; col < v_size_; col++)
            data_M[row*v_size_ + col] = 0;
      	}
        for (int l = block_row*v_size_; l < m1_size_; l++)
            data_M[l] = 0;

        // 2) Assign a m2
        for (int row = 0; row < block_col_1; row++) {
          for (int col = 0; col < block_col_2; col++)
            data_M[m1_size_ + row*v_size_ + col] = input_mat[(j+row)*num_matrix2 + (k+col)];
          for (int col = block_col_2; col < v_size_; col++)
            data_M[m1_size_ + row*v_size_ + col] = 0;
      	}
        for (int l = block_col_1*v_size_; l < m2_size_; l++)
            data_M[m1_size_ + l] = 0;

        // 3) Call a function `blockMM() to execute Matrix matrix multiplication
        const float* ret = this->blockMM();

        // 4) Accumulate intermediate results
        for(int n = 0; n<block_row; ++n)
        {
          for(int m = 0; m<block_col_2; ++m)
          {
            output[(i + n) + (k + m)*num_output] += ret[n*v_size_ + m];
          }
        }
      }
    } 
  }
}
\end{lstlisting}

\begin{itemize*}
\item Block OP를 수행할 부분인 Weight Matrix를 \texttt{data\_M}에 먼저 넣는다 (24 라인).
\item 이 행렬의 크기는 \texttt{block\_row} * \texttt{block\_col\_1} 이므로 첫 번째 행렬 부분에 해당하는 \texttt{m1\_size\_} 중 사용하지 않는 부분은 모두 0으로 초기화한다 (26-29 라인).
\begin{itemize*}
\item 그렇지 않다면 전 step에서 사용되었던 벡터 값이 잘못된 연산 결과를 초래할 수 있기 때문이다~\cite{lab2}.
\end{itemize*}
\item 다음으로 Input Matrix를 \texttt{data\_M}에 넣는다 (34 라인). 
\item 이 행렬의 크기는 \texttt{block\_col\_2} * \texttt{block\_col\_1} 이므로 두 번째 행렬 부분에 해당하는 \texttt{m2\_size\_} 중 사용하지 않는 부분은 모두 0으로 초기화한다 (36-39 라인).
\end{itemize*}

\newpage
\subsection{Convolution Lowering}

\begin{figure}[htb!]
	\centering
	\includegraphics[width=0.9\textwidth]{fig/convlow.jpg}
\caption{Convolution Filter와 Image data에 해당하는 3차원 텐서를 행렬화 시키기 위한 과정을 도식화 한 것~\cite{chetlur2014cudnn, lab9}. 여기서 Convolution Lowering은 \textit{inputs}와 \textit{weight}를 각각 \textit{new\_inputs}와 \textit{new\_weight}로 매핑하는 과정이다.}
\label{fig4}
\end{figure}

Figure~\ref{fig4}를 참조하면 \textit{new\_weight}의 행들은 \textit{weight}의 \texttt{channel}로 나눠지고 각 행은 \texttt{input\_channel} 순서로 \textit{weight} 값이 순서대로 나온다. 아래의 코드의 30-31 라인처럼 \texttt{conv\_channel}, \texttt{input\_channel}, \texttt{conv\_height}, \texttt{conv\_width} 순서대로 for문을 돌면서 \textit{new\_weight} 원소 값을 채워준다. \\

\textit{new\_weight}와 마찬가지로 \textit{new\_inputs}를 구성할 수 있다. 한 Filter가 Input에 방문하는 횟수는 행으로는 \linebreak$\texttt{input\_height} – \texttt{conv\_height} + 1$이고 열의 방향으로는 $\texttt{input\_width} – \texttt{conv\_width} + 1$이 될 것이다. 순서대로 for문을 돌면서 \textit{new\_inputs}의 행열 순서로 채워간다 (38-39 라인).

\subsection*{\texttt{FPGA::convLowering}}
\begin{lstlisting}[style={c-style}]
void FPGA::convLowering(const std::vector<std::vector<std::vector<std::vector<float>>>>& cnn_weights,
    std::vector<std::vector<float>>& new_weights,
    const std::vector<std::vector<std::vector<float>>>& inputs,
    std::vector<std::vector<float>>& new_inputs) {
  /*
   * Arguments:
   *
   * conv_weights: [conv_channel, input_channel, conv_height, conv_width]
   * new_weights: [conv_channel, input_channel*conv_height*conv_width]
   * inputs: [input_channel, input_height, input_width]
   * new_inputs: [input_channel*conv_height*conv_width, (input_height-conv_height+1)*(input_width-conv_width+1)]
   *
   */

  int conv_channel = cnn_weights.size();
  int input_channel = cnn_weights[0].size();
  int conv_height = cnn_weights[0][0].size();
  int conv_width = cnn_weights[0][0][0].size();
  //int input_channel = cnn_weights.size();
  int input_height = inputs[0].size();
  int input_width = inputs[0][0].size();

  // For example,
  // new_weights[0][0] = cnn_weights[0][0][0][0];
  // new_inputs[0][0] = inputs[0][0][0];
  for (int i = 0; i < conv_channel; i++)
    for (int j = 0; j < input_channel; j++)
      for (int k = 0; k < conv_height; k++)
        for (int l = 0; l < conv_width; l++)
          new_weights[i][j*conv_height*conv_width + k*conv_width + l] 
            = cnn_weights[i][j][k][l];

  for (int i = 0; i < input_channel; i++)
    for (int j = 0; j < conv_height; j++)
      for (int k = 0; k < conv_width; k++)
        for (int l = 0; l < input_height-conv_height+1; l++)
          for (int m = 0; m < input_width-conv_width+1; m++)
            new_inputs[i*conv_height*conv_width + j*conv_width + k][l*(input_width-conv_width+1) + m] 
              = inputs[i][j+l][k+m];

}
\end{lstlisting}
\newpage
\section{Result}
구현한 코드를 CPU 상에서 Pre-trained MLP와 CNN network를 사용하여 정확도를 측정해보았다. MLP는 0.97의 정확도를 보였으며 CNN은 MLP보다 높은 0.98에서 1.0의 정확도를 보였다. 아래의 Figure~\ref{fig5}는 Lab 9에서 주어진 \texttt{benchmark.sh}를 수행하였을 때의 결과이다. \\

\begin{figure}[htb!]
	\centering
	\includegraphics[width=0.8\textwidth]{fig/benchmark results.png}
\caption{Zedboard CPU/FPGA상에서 Pre-trained MLP와 CNN를 사용해 MNIST 데이터셋에 대하여 test 한 결과. \texttt{benchmark.sh}에 포함되지 않은 FPGA CNN은 포함하지 않았다.}
\label{fig5}
\end{figure}

수행 성능에 영향을 미치는 환경변수는 \textit{m\_size},  \textit{v\_size}, \textit{num\_test\_images}가 있으며 네트워크의 종류 또한 변화시키면서 실험을 진행하였다. 기본 값으로는 $\textit{m\_size}=\textit{v\_size}=16$과 $\textit{num\_test\_images}=100$으로 설정하였다. 관측 변수로는 정확도 \textit{accuracy}, 걸리는 시간 \textit{total\_time}, \textit{avg\_num\_call}을 측정하였다. 

\newpage
\begin{table}[htb!]
\renewcommand{\arraystretch}{1.1}
\begin{center}
\begin{tabular}{ |c | c | rrr |} 
 \hline
 Network & Control Variable & \textit{total\_time} & \textit{avg\_num\_call} & \textit{accuracy}  \\ 
 \hline
cnn & num\_test\_images(1) & 0.007s & 553 & 1 \\
cnn & num\_test\_images(10) & 0.067s & 553 & 1 \\
cnn & num\_test\_images(100) & 0.661s & 553 & 1 \\
cnn & num\_test\_images(1000) & 6.635s & 553 & 0.98 \\
cnn & num\_test\_images(10000) & 66.534s & 553 & 0.98 \\
 \hline
cnn & v\_size(1) & 1.589s & 44646 & 1 \\
cnn & v\_size(2) & 0.799s & 9141 & 1 \\
cnn & v\_size(4) & 0.619s & 3050 & 1 \\
cnn & v\_size(8) & 0.56s & 1188 & 1 \\
cnn & v\_size(16) & 0.662s & 553 & 1 \\
cnn & v\_size(32) & 1.34s & 277 & 1 \\
cnn & v\_size(64) & 4.042s & 140 & 1 \\
cnn & v\_size(128) & 37.86s & 71 & 1 \\
 \hline
mlp & num\_test\_images(1) & 0.068s & 9375 & 1 \\
mlp & num\_test\_images(10) & 0.675s & 9375 & 0.9 \\
mlp & num\_test\_images(100) & 6.738s & 9375 & 0.97 \\
mlp & num\_test\_images(1000) & 67.441s & 9375 & 0.92 \\
mlp & num\_test\_images(10000) & 675.352s & 9375 & 0.9159 \\
 \hline
mlp & v\_size(1) & 13.591s & 150000 & 0.97 \\
mlp & v\_size(2) & 9.524s & 75000 & 0.97 \\
mlp & v\_size(4) & 7.34s & 37500 & 0.97 \\
mlp & v\_size(8) & 6.323s & 18750 & 0.97 \\
mlp & v\_size(16) & 6.743s & 9375 & 0.97 \\
mlp & v\_size(32) & 6.537s & 4763 & 0.97 \\
mlp & v\_size(64) & 6.398s & 2419 & 0.97 \\
mlp & v\_size(128) & 6.562s & 1285 & 0.97 \\
 \hline
mlp & m\_size(1) & 8.767s & 149550 & 0.97 \\
mlp & m\_size(2) & 7.663s & 74775 & 0.97 \\
mlp & m\_size(4) & 7.104s & 37425 & 0.97 \\
mlp & m\_size(8) & 6.83s & 18750 & 0.97 \\
mlp & m\_size(16) & 6.745s & 9375 & 0.97 \\
mlp & m\_size(32) & 6.789s & 4787 & 0.97 \\
mlp & m\_size(64) & 6.957s & 2431 & 0.97 \\
mlp & m\_size(128) & 7.502s & 1315 & 0.97 \\
 \hline
\end{tabular}
\caption{ $\textit{m\_size}=\textit{v\_size}=16$과 $\textit{num\_test\_images}=100$로 설정하고 매개 변수를 하나씩 바꿔가면서 측정한 연산 성능 \textit{accuracy}, \textit{total\_time}, \textit{avg\_num\_call} 비교}\label{tab1}
\end{center}
\end{table}

\newpage
\section{Conclusion}
이번 실습에서는 Convolution Lowering을 사용해 CNN 연산을 Matrix Multiplication 연산으로 바꿔보았다. 또한 Lab 2에서 구현한 Matrix Vector Multiplication도 불러와 함께 연산에 사용하였다. MNIST 데이터에 대해서 각각 MLP, CNN으로 inference 한 후 비교해 본 결과 CNN 결과가 조금 더 높은 정확도를 보였다. \\

Convolution Filter의 Weights과 Inputs을 2차원 Matrix를 바꾸는 과정에서 같은 값이 여러 번 사용되어 행렬을 구성하는 것을 확인할 수 있었다. 다시 말하면 더 많은 Memory Allocation을 필요로 하였고 이 부분에서 더 최적화 가능할 것이라고 판단하였다. 또한 실제로 구현되는 CNN을 보면 매개변수로 Stride나 Padding을 넘겨줄 수 있다. 이번 Lab 9에는 이 부분이 빠져 있지만 추가된다면 더욱 더 일반적인 네트워크를 위한 Convolution Lowering을 구현할 수 있을 것이다. \\

\bibliographystyle{plain}
\bibliography{other}

\end{document}
