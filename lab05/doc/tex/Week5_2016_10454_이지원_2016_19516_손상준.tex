\documentclass{article}
\usepackage{bm}
\usepackage{amsmath}
\usepackage{graphicx}
\usepackage{mdwlist}
\usepackage[colorlinks=true]{hyperref}
\usepackage{geometry}
\usepackage{kotex}
\geometry{margin=1in}
\geometry{headheight=2in}
\geometry{top=2in}
\usepackage{palatino}
%\renewcommand{\rmdefault}{palatino}
\usepackage{fancyhdr}

%\pagestyle{fancy}
\rhead{}
\lhead{}
\chead{%
  {\vbox{%
      \vspace{2mm}
      \large
      Hardware System Design 4190.309A\hfill
\\
      Seoul National University
      \\[4mm]
      \textbf{Practice \#5. PE implementation \& BRAM modeling}\\
      \textbf{Jiwon Lee, Sangjun Son}
    }
  }
}

%%%%%%%%%%%%%%%%%%%%%%%
\usepackage{xcolor}
\usepackage{listings}
\definecolor{vgreen}{RGB}{104,180,104}
\definecolor{vblue}{RGB}{49,49,255}
\definecolor{vorange}{RGB}{255,143,102}

\lstdefinestyle{verilog-style}
{
    language=Verilog,
    basicstyle=\scriptsize\ttfamily,
    keywordstyle=\color{vblue},
    identifierstyle=\color{black},
    commentstyle=\color{vgreen},
    numbers=left,
    numberstyle=\tiny\color{black},
    numbersep=10pt,
    tabsize=8,
    moredelim=*[s][\colorIndex]{[}{]},
    literate=*{:}{:}1
}

\makeatletter
\newcommand*\@lbracket{[}
\newcommand*\@rbracket{]}
\newcommand*\@colon{:}
\newcommand*\colorIndex{%
    \edef\@temp{\the\lst@token}%
    \ifx\@temp\@lbracket \color{black}%
    \else\ifx\@temp\@rbracket \color{black}%
    \else\ifx\@temp\@colon \color{black}%
    \else \color{vorange}%
    \fi\fi\fi
}
\makeatother

\usepackage{trace}
%%%%%%%%%%%%%%%%%%%%%%%

\usepackage{paralist}

\usepackage{todonotes}
\setlength{\marginparwidth}{2.15cm}

\usepackage{tikz}
\usetikzlibrary{positioning,shapes,backgrounds}

\begin{document}

\pagestyle{fancy}

\section*{Goal}

\begin{itemize*}
\item Implement BRAM model \& test bench according to scenarios.
\begin{itemize*}
\item Make test bench that instantiates two BRAMs and initialize one BRAM to store address as data.
\item Copy every data from the initialized BRAM to the other BRAM.
\end{itemize*}
\item Implement PE with floating point fused multiply adder.
\end{itemize*}

\section{Implementation}

이번 프로젝트는 Block Random Access Memory와 Processing Element를 각각 구현함으로써 추후 구현할 Matrix-Matrix Multiplication을 수행하기 위한 기본 모듈을 구성하는 것을 목적으로 한다. 아래는 코드 구현과 함께 간략한 아이디어 및 기능에 대한 설명이 (1) BRAM, (2) PE 순으로 진행된다.

\subsection{Block Random Access Memory, BRAM}
BRAM의 경우 크게 두 부분으로 이뤄져 있다. (1) 모듈이 실행이 되기 전 INIT\_FILE에서 내부 메모리 \texttt{mem}를 읽는 부분과 done 신호가 주어졌을 때, OUT\_FILE에 \texttt{mem}의 상태를 출력하는 부분과, (2) EN, RST, WE 신호가 들어왔을 때 경우에 따라 입력되는 데이터를 \texttt{mem}에 읽고 쓰는 역할을 한다.

\begin{itemize*}
\item 아래에 첨부된 코드 중 21-28 라인의 외부 파일 입출력에 관한 구현 보면, \texttt{initial} 구문을 이용하여 모듈의 생성과 동시에 파일 입출력에 대한 실행 구문에 대한 scope를 지정한다. 
\texttt{\$readmemh} 로 시작함으로써 INIT\_FILE 파일을 읽어 \texttt{mem}에 저장한다. 
그 후 \texttt{done} 신호 들어올 때 까지 대기하다가 신호가 들어오면 \texttt{\$writememh} 함수를 사용해 OUT\_FILE에 \texttt{mem} 데이터를 저장한다.
이 때 함수에 붙어있는 \texttt{\$readmem\textbf{h}}와 \texttt{\$writemem\textbf{h}}의 \textbf{h}는 hexadecimal로 파일에 저장하는 값을 16진수 형태로 저장하는 옵션을 의미한다~\cite{memh}. \\
\item 30-46 라인은 BRAM의 input으로 주어지는 신호에 따라 모듈로써의 기능을 구현하는 부분이다. BRAM\_CLK와 BRAM\_RST 그리고 BRAM\_EN, BRAM\_WE의 신호에 따라 읽기, 쓰기, 초기화, 파일 출력을 위한 기능을 수행하게 된다. BRAM\_RST은 BRAM\_CLK에 Async로, BRAM\_EN, BRAM\_WE는 Sync로 구현하였다.
\begin{itemize*}
\item BRAM\_RST이 posedge일 경우 BRAM\_RDDATA에는 0을 할당한다. (31 라인)
\item BRAM\_EN이 활성화되어 있고 BRAM\_WE 또한 활성화되어 있다면, True를 가지는 bit에 해당하는 영역을 BRAM\_WRDATA에서 \texttt{mem}으로 복사한다. (35-38 라인)
\begin{equation}
\texttt{mem}[ \texttt{addr} ][8*(i+1)-1:8* i ]  \leftarrow \text{BRAM\_WRDATA}[8*(i+1) -1:8* i ] 
\end{equation}
\item BRAM\_EN이 활성화되어 있고 BRAM\_WE이 활성화되어 있다면, 메모리로부터 데이터를 읽어오는 기능을 수행한다. Read에 걸리는 싸이클이 2 cycle이 걸리도록 구현을 해야하기 때문에 \texttt{dout}을 버퍼로 사용해 1 cycle이 추가되도록 한다. (41-42 라인)
\end{itemize*}
\end{itemize*}

\newpage
\subsubsection*{\texttt{MY\_BRAM}}
\begin{lstlisting}[style={verilog-style}]
`timescale 1ns / 1ps
module my_bram #(
    parameter integer BRAM_ADDR_WIDTH = 15,
    parameter INIT_FILE = "input.txt",
    parameter OUT_FILE = "output.txt"
)(
    input wire [BRAM_ADDR_WIDTH-1:0] BRAM_ADDR,
    input wire BRAM_CLK,
    input wire [31:0] BRAM_WRDATA,
    output reg [31:0] BRAM_RDDATA,
    input wire BRAM_EN,
    input wire BRAM_RST,
    input wire [3:0] BRAM_WE,
    input wire done
);
    reg [31:0] mem[0:8191];
    wire [BRAM_ADDR_WIDTH-3:0] addr = BRAM_ADDR[BRAM_ADDR_WIDTH-1:2];
    reg [31:0] dout;
    
    initial begin
        if (INIT_FILE != "") begin
            $readmemh(INIT_FILE, mem);
        end
        wait (done) begin
            $writememh(OUT_FILE, mem);
        end
    end
    
    always @(posedge BRAM_CLK or posedge BRAM_RST) begin
        if (BRAM_RST) begin 
            BRAM_RDDATA <= 0;
        end
        if (BRAM_EN) begin
            if (BRAM_WE) begin
                if (BRAM_WE[0]) mem[addr][7:0] <= BRAM_WRDATA[7:0];
                if (BRAM_WE[1]) mem[addr][15:8] <= BRAM_WRDATA[15:8];
                if (BRAM_WE[2]) mem[addr][23:16] <= BRAM_WRDATA[23:16];
                if (BRAM_WE[3]) mem[addr][31:24] <= BRAM_WRDATA[31:24]; 
            end
            else begin
                dout <= mem[addr];
                BRAM_RDDATA <= dout;
            end
        end
    end
endmodule
\end{lstlisting}

\newpage
\subsection{Processing Element, PE}

아래의 모듈은 Lab03 adder 모듈이다. ain과 bin 길이의 입력값이 

\subsubsection*{\texttt{MY\_PE}}
\begin{lstlisting}[style={verilog-style}]
`timescale 1ns / 1ps

module my_pe #(
    parameter L_RAM_SIZE = 6,
    parameter BITWIDTH = 32
)
(
    input aclk,
    input aresetn,
    input [BITWIDTH-1:0] ain,
    input [BITWIDTH-1:0] bin,
    input valid,
    output dvalid,
    output [BITWIDTH-1:0] dout
);

    // local reg ( can make overflow )
    reg [BITWIDTH-1:0] psum = 0;
    wire [BITWIDTH-1:0] res;
    
    floating_point_MAC UUT (
        .aclk(aclk),
        .aresetn(aresetn),
        .s_axis_a_tvalid(valid),
        .s_axis_b_tvalid(valid),
        .s_axis_c_tvalid(valid),
        .s_axis_a_tdata(ain),
        .s_axis_b_tdata(bin),
        .s_axis_c_tdata(psum),
        .m_axis_result_tvalid(dvalid), 
        .m_axis_result_tdata(res)
    );
  
    always @(dvalid) begin
        if(dvalid == 1) begin
            psum = res;
        end
    end
    
    assign dout = dvalid == 1 ? psum : 0;
endmodule
\end{lstlisting}

\newpage
\section{Result}

\subsection{Block Random Access Memory, BRAM}
아래의 코드는 BRAM의 구현의 Validity를 확인하기 위해 시나리오에 맞게 구현한 것이다. BRAM 인스턴스 두 개를 만들고 각각을 MY\_BRAM1과 MY\_BRAM2로 명명하였다. 

MY\_BRAM1은 \texttt{input.txt}에 저장된 \texttt{mem}에 있는 값들을 호출하여 저장하는 역할을 하고 또한 \texttt{mem}에 있는 값들을 MY\_ADDR을 변화하면서 BRAM\_RDDATA1로 읽어온다. MY\_BRAM2의 경우 이렇게 읽어온 \linebreak BRAM\_RDDATA1을 BRAM\_WRDATA2로 사용하여 \texttt{mem}에 저장하게 되고 완료가 되면 \texttt{done} 신호를 주어 \linebreak \texttt{output.txt}에 저장하게 된다. 아래 Figure~\ref{fig1}은 상기된 설명을 도식화한 것이다.
\begin{figure}[ht]
	\centering
	\includegraphics[width=0.49\textwidth]{fig/fig1.png}
	\includegraphics[width=0.49\textwidth]{fig/fig2.png}
\caption{Testbench Scenarios: 2개의 BRAM을 인스턴스화 하고 BRAM1에서 \texttt{input.txt}을 읽어 메모리에 저장하고 다른 BRAM2에서 BRAM1의 데이터를 전달받아  \texttt{output.txt}에 저장한다~\cite{lab}. }
\label{fig1}
\end{figure}
\begin{itemize*}
\item 모든 테스트를 시작하기 전에 \texttt{input.txt}를 초기화하기 위한 과정을 거친다. 주소에 해당하는 인덱스를 값으로 가질 수 있도록 for문을 통해 대입시킨 후 \texttt{\$writememh}를 통해 파일에 저장을 한다 (23-24 라인).\\
\item 테스트벤치에서 BRAM\_EN 신호는 항상 활성화 하고 (45, 55 라인) BRAM\_RST와 \texttt{done} 신호는 모든 데이터 전송이 끝나고 입력과 출력이 완료되었을 때 True를 대입할 것이다 (33-34 라인).\\
\item BRAM\_ADDR의 경우 $i$번째 entry의 주소값은 BRAM에서 2개의 LSB를 사용하지 않으므로 주소값 또한 4의 배수로 증가시켜 대입해 주어야 한다. BRAM\_WE 신호를 주소값이 유지되는 한 구간을 5 CLK 싸이클과 1 CLK 싸이클로 나누어 DISABLE과 ENABLE을 번갈아 대입해준다 (29-31 라인).
\end{itemize*}

\newpage
\subsubsection*{\texttt{TB\_MY\_BRAM}}
\begin{lstlisting}[style={verilog-style}]
`timescale 1ns / 1ps

module tb_my_bram #(
    parameter integer BRAM_ADDR_WIDTH = 15,
    parameter INIT_FILE = "input.txt"
)();
    reg [31:0] BRAM_INIT[0:8191];
    reg [BRAM_ADDR_WIDTH-1:0] BRAM_ADDR;
    reg BRAM_CLK;
    reg BRAM_RST;
    reg [3:0] BRAM_WE;
    reg done;
    wire [31:0] BRAM_WRDATA1, BRAM_RDDATA1;
    wire [31:0] BRAM_WRDATA2, BRAM_RDDATA2;
    integer i;
    
    initial begin
        BRAM_ADDR <= 0;
        BRAM_CLK <= 1;
        BRAM_RST <= 0;
        BRAM_WE <= 0;
        done <= 0;
        for (i = 0; i < 8192; i = i + 1) begin
            BRAM_INIT[i][31:0] <= i;
        end
        #10 $writememh(INIT_FILE, BRAM_INIT);
        
        for (i = 0; i <= 8192; i = i + 1) begin
            BRAM_ADDR <= i << 2; #20;
            BRAM_WE <= 4'b1111; #10;
            BRAM_WE <= 0; #30;
        end
        done <= 1'b1; #30;
        BRAM_RST <= 1'b1;
    end
    
    always #5 BRAM_CLK = ~BRAM_CLK;
    assign BRAM_WRDATA2 = BRAM_RDDATA1;
    
    my_bram MY_BRAM1 (
        .BRAM_ADDR(BRAM_ADDR),
        .BRAM_CLK(BRAM_CLK),
        .BRAM_WRDATA(BRAM_WRDATA1),
        .BRAM_RDDATA(BRAM_RDDATA1),
        .BRAM_EN(1'b1),
        .BRAM_RST(BRAM_RST),
        .BRAM_WE(0),
        .done(0)
    );
    my_bram #(.INIT_FILE("")) MY_BRAM2 (
        .BRAM_ADDR(BRAM_ADDR),
        .BRAM_CLK(BRAM_CLK),
        .BRAM_WRDATA(BRAM_WRDATA2),
        .BRAM_RDDATA(BRAM_RDDATA2),
        .BRAM_EN(1'b1),
        .BRAM_RST(BRAM_RST),
        .BRAM_WE(BRAM_WE),
        .done(done)
    );
endmodule
\end{lstlisting}

위 Testbench 코드를 수행하면 아래의 Figure~\ref{fig2}와 같은 Waveform을 확인할 수 있다. 결과를 자세히 보면 \texttt{mem}에 해당하는 BRAM\_ADDR이 4씩 증가하는 것을 확인할 수 있고 이에 따라 BRAM\_RDDATA1 또한 BRAM\_WE가 비활성화 되어 있을 때 2 cycle을 delay로 읽게 되는 것을 확인할 수 있다. BRAM\_RDDATA1가 곧 BRAM2의 BRAM\_WRDATA2이므로 같은 Waveform을 관찰하였다. \\

\texttt{mem}에 write가 될 때 1 cycle delay가 되는 것을 눈으로 확인할 수는 없지만 BRAM\_WE가 ENABLE 되었다가 DISABLE 되었을 때 read하는 데이터 BRAM\_RDDATA2의 delay가 총 3 cycle가 걸렸다는 것을 확인할 수 있었다. 이 사실로 미뤄 보아 읽기에 걸리는 시간이 2 CLK 싸이클, 쓰기에 걸리는 시간이 1 CLK 싸이클이 걸린다는 것을 유추할 수 있었고 스펙에 맞는 올바른 구현이 되었다는 것을 짐작할 수 있었다.\\

Figure~\ref{fig3}은 생성된 입력과 출력파일로 각각 \texttt{input.txt}는 MY\_BRAM1이 입력을 받기 위한 파일, \texttt{output.txt}는 MY\_BRAM2이 MY\_BRAM1으로부터 데이터를 전달받아 출력을 하기 위해 생성된 파일이다. 모든 데이터가 입력을 받은 대로 정상적으로 출력이 되었음을 확인하였다.
\begin{figure}[ht]
	\centering
	\includegraphics[width=1.0\textwidth]{../../submission/MY_BRAM/MY_BRAM_Waveform1.png}
	\includegraphics[width=1.0\textwidth]{../../submission/MY_BRAM/MY_BRAM_Waveform2.png}
\caption{\texttt{TB\_MY\_BRAM} Waveform}
\label{fig2}
\end{figure}
\begin{figure}[ht]
	\centering
	\includegraphics[width=0.6\textwidth]{../../submission/MY_BRAM/MY_BRAM_result.png}
\caption{\texttt{TB\_MY\_BRAM} I/O files. \texttt{input.txt}와 \texttt{output.txt}의 8192개의 \texttt{mem} 값이 서로 같다는 결과를 토대로 데이터 전달 및 입력/출력이 잘 이뤄진 것을 짐작할 수 있다.}
\label{fig3}
\end{figure}

\newpage
\subsection{Processing Element, PE}
\subsubsection*{\texttt{TB\_MY\_PE}}
\begin{lstlisting}[style={verilog-style}]
`timescale 1ns / 1ps

module tb_my_pe #(
    parameter L_RAM_SIZE = 6,
    parameter BITWIDTH = 32
);

    reg [BITWIDTH-1:0] gb1 [0:2**L_RAM_SIZE-1];
    reg [BITWIDTH-1:0] gb2 [0:2**L_RAM_SIZE-1];
    reg [BITWIDTH-1:0] ain;
    reg [BITWIDTH-1:0] bin;
    reg rst;
    reg clk;
    wire [BITWIDTH-1:0] dout;
    reg valid;
    wire dvalid;
   
    integer i;         
    initial begin
        clk <= 0;
        rst <= 1;
        #20;
        rst <= 0;
        
        // value setting in global buffers
        for(i = 0; i < 16; i = i+1) begin
            gb1[i] = $urandom_range(2**30,2**30+2**26);
            gb2[i] = $urandom_range(2**30,2**30+2**26);
//            gb1[i] <= ($urandom%2 << 31) + ($urandom%(2**8) << 23) + $urandom%(2**23);
//            gb2[i] <= ($urandom%2 << 31) + ($urandom%(2**8) << 23) + $urandom%(2**23);
        end
        
        // execute PE
        for(i = 0; i < 16; i = i+1) begin
            if(i != 0) begin
                wait (dvalid == 1);
                wait (dvalid == 0);
            end
            ain = gb1[i];
            bin = gb2[i];
            #15;
            valid <= 1;
            #10;
            valid <= 0;
        end
    end
    
    always #5 clk = ~clk;
    
    my_pe MY_PE (
        .aclk(clk),
        .aresetn(~rst),
        .ain(ain),
        .bin(bin),
        .valid(valid),
        .dvalid(dvalid),
        .dout(dout)
    );
    
endmodule
\end{lstlisting}

위 Testbench 코드를 수행하면 아래의 Figure~\ref{fig3}와 같은 Waveform을 확인할 수 있다. 
\begin{figure}[ht]
	\centering
	\includegraphics[width=1.0\textwidth]{../../submission/MY_PE/MY_PE_Waveform1.png}
	\includegraphics[width=1.0\textwidth]{../../submission/MY_PE/MY_PE_Waveform2.png}
\caption{\texttt{TB\_MY\_PE} Waveform}
\label{fig3}
\end{figure}

\newpage
\section{Conclusion}

이후 프로젝트에서 어떤 모듈을 구현해야 하는지 Bottom-up으로 구현하다 보니 무슨 기능을 위한 구현인지는 아직 잘 모르겠지만 반대로 이전 lab 세션에서 구현을 진행한 모듈에 대해서는 연계성을 확인할 수 있었다. 지금 구현한 모듈이 앞으로도 쓰일 수 있기 때문에 가독성을 높이면서 최대한 임의 구현 방식을 최대한 피하기 위해 노력하였다. \\

MY\_BRAM 모듈을 구현하면서 WE signal에 따라 \texttt{mem}에 저장하는 statement를 for-generate로 구현해보려 했으나 이런 저런 오류가 나면서 나열형 방식으로 구현해 코드의 효율성이 떨어진다는 나름의 판단을 하였다. 추후 프로젝트를 진행하기 전에 always 구문 안에서 block assignment를 for-generate로 구현하는 방식을 익혀야 겠다는 필요성을 제고하였다~\cite{thomas2008verilog}.

\bibliographystyle{plain}
\bibliography{other}

\end{document}
